\documentclass[12pt]{article}
%% Insert Commands in Preface.tex
\addtolength{\hoffset}{-2.25cm}
\addtolength{\textwidth}{4.5cm}
\addtolength{\voffset}{-2.5cm}
\addtolength{\textheight}{5cm}
\setlength{\parskip}{0pt}
\setlength{\parindent}{15pt}

\usepackage{amsthm, amsmath, amssymb, mathbbol}
\usepackage[colorlinks = true, linkcolor = black, citecolor = black, final]{hyperref}

\newtheorem{theorem}{Theorem}[section]
\newtheorem{corollary}[theorem]{Corollary}
\newtheorem{lemma}[theorem]{Lemma}
\newtheorem{example}[theorem]{Example}
\newtheorem{definition}[theorem]{Definition}

\usepackage{mathrsfs}
\usepackage{graphicx}
\usepackage{multicol}
\usepackage{accents}
\usepackage{tikz}
\usepackage{xcolor}
\usetikzlibrary{patterns}

\setlength{\parindent}{0in}

\pagestyle{empty}

%% Insert Commands Below This Line


\begin{document}

\thispagestyle{empty}

{\scshape Huang Linhang; Syx Pek; Qi Ji} \hfill {\scshape \large Differential Geometry} \hfill {\scshape Homework \#2}
 
\smallskip
\hrule
\bigskip
\section{Section 2.5}
\subsection*{Question 1}

\subsection*{Question 3}

\subsection*{Question 5}

\subsection*{Question 7}

\subsection*{Question 9}

\subsection*{Question 11}

\section{Section 3.2}
\subsection*{Question 1}

\subsection*{Question 3}

\section{Section 3.3}

\subsection*{Question 1}

\subsection*{Question 3}
Determine the asymptotic curves of the catenoid
\begin{equation*}
    \mathbf{x}(u,v)= (\cosh{v}\cos{u},\cosh{v}\sin{u},v).
\end{equation*}

\textit{Solution.} We have \begin{align*}
    &\mathbf{x}_u= (-\cosh{v}\sin{u},\cosh{v}\cos{u},0),\\
    &\mathbf{x}_v= (\sinh{v}\cos{u},\sinh{v}\sin{u},1),\\
    &\mathbf{x}_{uu}= (-\cosh{v}\cos{u},-\cosh{v}\sin{u},0),\\
    &\mathbf{x}_{uv}= (-\sinh{v}\sin{u},\sinh{v}\cos{u},0),\\
    &\mathbf{x}_{vv}= (\cosh{v}\cos{u},\cosh{v}\sin{u},0).\\
\end{align*}
Hence, \begin{align*}
    N &= \frac{\mathbf{x}_u\wedge \mathbf{x}_v}{|\mathbf{x}_u\wedge \mathbf{x}_v|}\\
    &=\frac{(\cosh{v}\cos{u},\cosh{v}\sin{u},-\cosh{v}\sinh{v})}{(\cosh{v})^2}\\
    &=\frac{1}{\cosh{v}}(\cos{u},\sin{u},-\sinh{v}).
\end{align*}
Then\begin{align*}
    e &= \langle N,\mathbf{x}_{uu}\rangle = -1,\\
    f &= \langle N,\mathbf{x}_{uv}\rangle = 0,\\
    g &= \langle N,\mathbf{x}_{vv}\rangle = 1.\\
\end{align*}
This implies that the asymptotic directions correspond to $\langle u',v'\rangle$ satisfies \begin{align*}
    &e(u')^2 + 2fu'v' + g(v')^2 = 0\\
    \Rightarrow &-(u')^2 + (v')^2 = 0\\
    \Rightarrow & u'=v' \quad\text{or}\quad u'=-v'.
\end{align*}
This shows that the asymptotic curves are the traces of $v=u+a$ or $v=-u+b$ for some $a,b\in\mathbb{R}$, which are \begin{align*}
    &\alpha_a(u) = \mathbf{x}(u,u+a) = (\cosh{(u+a)}\cos{u},\cosh{(u+a)}\sin{u},u+a),\\
    &\beta_b(u) = \mathbf{x}(u,-u+b) = (\cosh{(-u+b)}\cos{u},\cosh{(-u+b)}\sin{u},-u+b).
\end{align*}
The collection of all $\alpha_a,\beta_b$ are all the asymptotic curves.

\subsection*{Question 9}
(Contact of Curves.) Define contact of order $\geq n$ ($n$ integer $\geq 1$) for regular curves in R3 with a common point $p$ and prove that\\
a. The notion of contact of order $\geq n$ is invariant by diffeomorphisms.\\
b. Two curves have contact of order $\geq 1$ at $p$ if and only if they are tangent at $p$.\\

\textit{Solution.} We say two surfaces $S$ and $\bar{S}$ with a common point $p$ to have contact of order $\geq n$ at $p$ if there exist parametrizations $\mathbf{x}(u, v)$ and $\mathbf{\tilde{x}}(u,v)$ in $p$ of $S$ and $\bar{S}$ such that the partial derivatives of $\mathbf{x}$ and $\mathbf{\tilde{x}}(u,v)$ agree up to order $n$.\\\\
a. Let $\psi:\mathbb{R}^3 \to \mathbb{R}^3$ be a diffeomorphism. Then for any partial derivative operator $\partial_I$ of order less than $n$ on $u-v$ space. Then \begin{equation*}
    \partial_I\psi(\mathbf{x}) = J_\psi(\mathbf{x})\partial_I\mathbf{x} = J_\psi(\mathbf{\tilde{x}})\partial_I\mathbf{\tilde{x}} = \partial_I\psi(\mathbf{\tilde{x}}),
\end{equation*}
where $J\psi(\cdot)$ is the Jacobian of $\psi$. This shows that the notion of contact is invariant by diffeomorphisms.\\\\
b. It is easy to see that the contact of order $\geq 1$ implies that the two surfaces are tangent. For the converse, we suppose $S$ and $\bar{S}$ are tangent at $p$ with parametrizations $\mathbf{x(u,v)}$ and $\mathbf{\tilde{x}}(u,v)$ respectively. Then at the point $p$, $\mathbf{\tilde{x}}_u, \mathbf{\tilde{x}}_v \in T_{\mathbf{x}}(p)$ we can write \begin{align*}
    &\mathbf{\tilde{x}}_u = a_1 \mathbf{x}_u + a_2 \mathbf{x}_v,\\
    &\mathbf{\tilde{x}}_v = b_1 \mathbf{x}_u + b_2 \mathbf{x}_v.\\
\end{align*}
Note that since $\mathbf{\tilde{x}}_u, \mathbf{\utilde{x}}_v$ are linearly independent, we have $a_1b_2-a_2b_1\neq 0$. Now let $w = \frac{b_2u-a_2v}{a_1b_2-a_2b_1}$, $l = \frac{b_1u-a_1v}{a_2b_1-a_1b_2}$ and $\mathbf{{y}}(w,l) = \mathbf{\tilde{x}}(u,v)$. Then \begin{align*}
    \mathbf{y}_w &= \frac{b_2}{a_1b_2-a_2b_1}\mathbf{\tilde{x}}_u-\frac{a_2}{a_1b_2-a_2b_1}\mathbf{\tilde{x}}_v=\mathbf{x}_u,\\
    \mathbf{y}_l &=
    \frac{b_1}{a_2b_1-a_1b_2}\mathbf{\tilde{x}}_u-\frac{a_1}{a_2b_1-a_1b_2}\mathbf{\tilde{x}}_v=\mathbf{x}_v.
\end{align*}
This shows that $S$ and $\bar{S}$ have contact of order $\geq 1$ at p. 

\subsection*{Question 15}
Give an example of a surface which has an isolated parabolic point $p$ (that is, no other parabolic point is contained in some neighborhood of $p$).

\textit{Solution.} Consider the graph $(x,y,x^4+x^2y^2+y^2)$. Let $h(x,y)=x^4+x^2y^2+y^2$. Then we have\begin{align*}
    K &= \frac{h_{xx}h_{yy}-(h_{xy})^2}{(1+h_x^2+h_y^2)^2} = \frac{24x^4-12x^2y^2+24x^2+4y^2}{(1+h_x^2+h_y^2)^2},\\
    e &= \frac{h_{xx}}{(1+h_x^2+h_y^2)^{1/2}}=\frac{12x^2+2y^2}{(1+h_x^2+h_y^2)^{1/2}},\\
    f &= \frac{h_{xy}}{(1+h_x^2+h_y^2)^{1/2}}=\frac{2x^2+2}{(1+h_x^2+h_y^2)^{1/2}},\\
    g &= \frac{h_{yy}}{(1+h_x^2+h_y^2)^{1/2}}=\frac{4xy}{(1+h_x^2+h_y^2)^{1/2}}.
\end{align*}
 Then $K=0$ only at $(0,0,0)$, at which $f$ is nonzero. This shows that the graph has an isolated parabolic point.n $K=0$ only at $(0,0,0)$, at which $f$ and $g$ are nonzero. This shows that the graph has an isolated parabolic point.
 
\subsection*{Question 19}
Obtain the asymptotic curves of the one-sheeted hyperboloid $x^2 + y^2 - z^2 = 1$.

\textit{Solution.} Note that the hyperboloid is a surface of revolution parametrized by \begin{equation*}
    \mathbf{x}(u,v) = (\phi(v)\cos{u},\phi(v)\sin{u},\psi(v)),
\end{equation*}
where $\phi(v) = \cosh{v},\psi(v)=\sinh{v}$ and $u\in(0,2\pi)$. Then\begin{align*}
    e &=  -\phi\psi'=-\cosh^2(v),\\
    f &= 0,\\
    g &= \psi'\phi''-\psi''\phi'=\cosh^2(v)-\sinh^2(v)=1.
\end{align*}
Then solving $e(u')^2 +2fu'v'+g(v')^2=0$, we have \begin{equation*}
    v'=u'\cosh(v) \qquad \text{or} \qquad v'=-u'\cosh(v).
\end{equation*}
Solving the ODE, we have \begin{equation*}
    u(t) = \pm \tan^{-1}(\sinh v(t)) + C,\qquad C\in \mathbb{R}
\end{equation*}
Hence, the asymptotic curves will be the trace of $\gamma_C(v) = (\pm\tan^{-1}( \sinh v)  + C,v), v\in R$. They are \begin{equation*}
    \alpha_C(v) = \mathbf{x}(\tan^{-1}( \sinh v)  + C,v)
\end{equation*} or \begin{equation*}
    \beta_C(v) = \mathbf{x}(-\tan^{-1}( \sinh v)  + C,v)
\end{equation*}

\subsection*{Question 21}
Let $S$ be a surface with orientation $N$. Let $V \subset S$ be an open set in $S$ and let $f : V \subset S \to R$ be any nowhere-zero differentiable function in $V$. Let $v_1$ and $v_2$ be two differentiable (tangent) vector fields in $V$ such that at each point of $V$, $v_1$ and $v_2$ are orthonormal and $v_1 \wedge v_2 = N$.\\
a. Prove that the Gaussian curvature $K$ of $V$ is given by \begin{equation*}
    K = \frac{\langle dfN(v_1)\wedge dfN(v_2),fN\rangle}{f^3}.
\end{equation*}
b. Apply the above result to show that iff is the restriction of \begin{equation*}
    \sqrt{\frac{x^2}{a^4}+\frac{y^2}{b^4}+\frac{z^2}{c^4}}
\end{equation*}
to the ellipsoid \begin{equation*}
    \frac{x^2}{a^2}+\frac{y^2}{b^2}+\frac{z^2}{c^2}=1,
\end{equation*}
 then the Gaussian curvature of the ellipsoid is\begin{equation*}
     K = \frac{1}{a^2b^2c^2}\frac{1}{f^4}.
\end{equation*}
 
\textit{Solution.} Since $f$ is a smooth function on $V=\mathbf{x}(u,v)$, if $\alpha'(0) = v_i = \frac{d}{dt}\mathbf{x}(\beta(t))|_{t=0}$, we have \begin{align*}
     dfN(v_i) &= \frac{d}{dt}f(\alpha(t))N(\alpha(t))\\
     &=(\frac{d}{dt}f(\alpha(t)))N(\alpha(t))|_{t=0} + f(\alpha(t))\frac{d}{dt}N(\alpha(t))|_{t=0}\\
     &=(\nabla(f\circ \mathbf{x}) \cdot \beta'(0))N + fdN(v_i).
\end{align*}
Hence, \begin{align*}
    &dfN(v_1) \wedge dfN(v_2) = (C_1N + fdN(v_1)) \wedge (C_2N + fdN(v_2))\\
    &= C_1N\wedge fdN(v_2) - C_2N \wedge fdN(v_1)+f^2(dN(v_1) \wedge dN(v_2))\\
    &= C_1N\wedge fdN(v_2) - C_2N \wedge fdN(v_1)+f^2\det(dN)(v_1 \wedge v_2)\\
    &= C_1N\wedge fdN(v_2) - C_2N \wedge fdN(v_1)+f^2\det(dN)N.
\end{align*}
Therefore, \begin{align*}
    dfN(v_1) \wedge dfN(v_2) \cdot fN &= C_1N\wedge fdN(v_2) \cdot fN - C_2N \wedge fdN(v_1) \cdot fN +f^2\det(dN)N \cdot fN\\
    &=f^3KN\cdot N = f^3K.
\end{align*}
Thus \begin{equation*}
    \frac{dfN(v_1) \wedge dfN(v_2) \cdot fN }{f^3}=K.
\end{equation*}
b. We know that \begin{align*}
    N(x,y,z) &= \frac{(\frac{2x}{a^2},\frac{2y}{b^2},\frac{2z}{c^2})}{|(\frac{2x}{a^2},\frac{2y}{b^2},\frac{2z}{c^2})|}\\
    &= \frac{(\frac{x}{a^2},\frac{y}{b^2},\frac{z}{c^2})}{|(\frac{x}{a^2},\frac{y}{b^2},\frac{z}{c^2})|}\\
    &=\frac{(\frac{x}{a^2},\frac{y}{b^2},\frac{z}{c^2})}{f(x,y,z)}.
\end{align*}
Therefore, $fN = (\frac{x}{a^2},\frac{y}{b^2},\frac{z}{c^2})$. Then \begin{align*}
    \frac{d}{dt}fN(\alpha(t))&= (\frac{x'(t)}{a^2},\frac{y'(t)}{b^2},\frac{z'(t)}{c^2})\\
    &=\begin{pmatrix} a^{-2}&&\\&b^{-2}&\\&&c^{-2}\end{pmatrix} \alpha'(t).
\end{align*}
Hence, $dfN(v_i) = \begin{pmatrix} a^{-2}&&\\&b^{-2}&\\&&c^{-2}\end{pmatrix}v_i$ and thus \begin{align*}
    K &= \frac{dfN(v_1) \wedge dfN(v_2) \cdot fN }{f^3}\\
    &= \det(dfN)\frac{(dfN^{-1})^T(v_1\wedge v_2) \cdot fN}{f^3}\\
    &=(abc)^{-2}\frac{(dfN^{-1})N \cdot fN}{f^3}\\
    &=(abc)^{-2}\frac{1}{f^3}\begin{pmatrix} a^{2}&&\\&b^{2}&\\&&c^{2}\end{pmatrix}\frac{(\frac{x}{a^2},\frac{y}{b^2},\frac{z}{c^2})}{f(x,y,z)} \cdot (\frac{x}{a^2},\frac{y}{b^2},\frac{z}{c^2})\\
    & = (abc)^{-2}\frac{1}{f^3}\frac{(x,y,z)}{f} \cdot (\frac{x}{a^2},\frac{y}{b^2},\frac{z}{c^2})\\
    &= \frac{1}{a^2b^2c^2}\frac{1}{f^4}
\end{align*}






\end{document}
