\documentclass[12pt]{article}
%% Insert Commands in Preface.tex
\addtolength{\hoffset}{-2.25cm}
\addtolength{\textwidth}{4.5cm}
\addtolength{\voffset}{-2.5cm}
\addtolength{\textheight}{5cm}
\setlength{\parskip}{0pt}
\setlength{\parindent}{15pt}

\usepackage{amsthm, amsmath, amssymb, mathbbol}
\usepackage[colorlinks = true, linkcolor = black, citecolor = black, final]{hyperref}

\newtheorem{theorem}{Theorem}[section]
\newtheorem{corollary}[theorem]{Corollary}
\newtheorem{lemma}[theorem]{Lemma}
\newtheorem{example}[theorem]{Example}
\newtheorem{definition}[theorem]{Definition}

\usepackage{mathrsfs}
\usepackage{graphicx}
\usepackage{multicol}
\usepackage{accents}
\usepackage{tikz}
\usepackage{xcolor}
\usetikzlibrary{patterns}

\setlength{\parindent}{0in}

\pagestyle{empty}

%% Insert Commands Below This Line


\begin{document}

\thispagestyle{empty}

{\scshape Huang Linhang; Syx Pek; Qi Ji} \hfill {\scshape \large Differential Geometry} \hfill {\scshape Homework \#7}
 
\smallskip
\hrule
\bigskip

\section*{Section 4.3}
\subsection*{Question 2}

\newcommand*{\pdd}[2]{\frac{\partial #1}{\partial #2}}
\let\phi\varphi

Show that if \(\mathbf{x}\) is an isothermal parametrization, that is,
\(E = G = \lambda(u,v)\) and \(F = 0\), then
\[ K = -\frac{1}{2\lambda}\Delta(\log\lambda), \]
where \(\Delta\phi\) denotes the Laplacian
\(\pdd{^2\phi}{u^2} + \pdd{^2\phi}{v^2}\) of the function \(\phi\).
Conclude that when \(E = G = (u^2+v^2+c)^{-2}\) and \(F = 0\), then \(K = \text{const.} = 4c\).

\emph{Proof.}
As \(F = 0\), \(\mathbf{x}\) is orthogonal, so we can apply the formula from the previous exercise, which is
\[ K = -\frac{1}{2\sqrt{EG}} \left\{
        \left( \frac{E_v}{\sqrt{EG}} \right)_v +
        \left( \frac{E_u}{\sqrt{EG}} \right)_u
\right\}. \]
Note that \(\sqrt{EG} = \lambda\), also
\begin{align*}
    \left( \frac{E_v}{\sqrt{EG}} \right)_v + \left( \frac{E_u}{\sqrt{EG}} \right)_u
    &= \left( \frac{\lambda_v}{\lambda} \right)_v + \left( \frac{\lambda_u}{\lambda} \right)_u \\
    &= \left( \pdd{\log\lambda}{v} \right)_v + \left( \pdd{\log\lambda}{u} \right)_u \\
    &= \Delta(\log \lambda)
\end{align*}
which completes the proof.

When \(E = G = \lambda = (u^2+v^2+c)^{-2}\) and \(F = 0\),
\begin{align*}
    \log\lambda &= -2 \log (u^2+v^2+c) \\
    \pdd{\log\lambda}{u} &= -4 \frac{u}{u^2+v^2+c} \\
    \pdd{^2\log\lambda}{u^2} &= -4 \frac{u^2 + v^2 + c - 2u^2}{(u^2+v^2+c)^2} \\
                             &= -4\lambda (-u^2 + v^2 + c) \\
    \pdd{^2\log\lambda}{v^2} &= -4\lambda ( u^2 - v^2 + c) \\
    \Delta(\log\lambda) &= -4\lambda(2c) = -8c\lambda \\
    K &= -\frac{1}{2\lambda}\Delta(\log\lambda) = 4c.
\end{align*}

\subsection*{Question 3}

Verify that the surfaces
\begin{align*}
    \mathbf{x}(u,v) &= (u\cos v, u\sin v, \log u), \\
    \overline{\mathbf{x}}(u,v) &= (u\cos v, u\sin v, v),
\end{align*}
have equal Gaussian curvature at the points \(\mathbf{x}(u,v)\) and \(\overline{\mathbf{x}}(u,v)\)
but that the mapping \(\overline{\mathbf{x}} \circ \mathbf{x}^{-1}\) is not an isometry.
This shows that the ``converse'' of the Gauss theorem is not true.

\emph{Solution.}
We compute
\begin{align*}
    \mathbf{x}_u &= \left( \cos v, \sin v, \frac1u \right) \\
    \mathbf{x}_v &= \left( -u \sin v, u\cos v, 0 \right) \\
    E &= \left\langle \mathbf{x}_u, \mathbf{x}_u \right\rangle \\
      &= 1 + \frac1{u^2} \\
    F &= \left\langle \mathbf{x}_u, \mathbf{x}_v \right\rangle \\
      &= 0 \\
    G &= \left\langle \mathbf{x}_v, \mathbf{x}_v \right\rangle \\
      &= u^2 \\
    \overline{\mathbf{x}}_u &= \left( \cos v, \sin v, 0 \right) \\
    \overline{\mathbf{x}}_v &= \left( -u \sin v, u\cos v, 1 \right) \\
    \overline{E} &= \left\langle \overline{\mathbf{x}}_u, \overline{\mathbf{x}}_u \right\rangle \\
                 &= 1 \\
    \overline{F} &= \left\langle \overline{\mathbf{x}}_u, \overline{\mathbf{x}}_v \right\rangle \\
                 &= 0 \\
    \overline{G} &= \left\langle \overline{\mathbf{x}}_v, \overline{\mathbf{x}}_v \right\rangle \\
                 &= u^2 + 1
\end{align*}
Now as \(F = \overline{F} = 0\), we can simplify the checking of \(K = \overline{K}\) by applying the formula in exercise 1.
We compute
\[ EG = u^2 + 1 = \overline{EG} \]
and
\[ E_v = 0 = \overline{E}_v \]
and
\[ G_u = 2u = \overline{G}_u \]
which shows the two surfaces have the same Gaussian curvature.

To see that \(\overline{\mathbf{x}} \circ \mathbf{x}^{-1}\) is not an isometry,
consider this curve
\[ \alpha(t) = \mathbf{x}(t,\pi) = (-t, 0, \log t),\quad 1<t<2 \]
note that
\[ \overline{\mathbf{x}} \circ \mathbf{x}^{-1} \circ \alpha(t) = \overline{\mathbf{x}}(t,\pi) = (-t, 0, \pi). \]
We can compute and check that \(\overline{\mathbf{x}} \circ \mathbf{x}^{-1}\) fails to preserve the arc length of this curve.


\subsection*{Question 7}

Does there exists a surface with $E = 1$, $F = 0$ and $G = \cos(u)^2$, and 
$e = \cos(u)^2$, $f = 0$ and $g = 1.$

\textit{Solution.} No. Observe that the Codazzi-Gauss-Mainardi equation states
$$M_v - N_u = \Gamma^1_{22}L + \Gamma^2_{22}M - \Gamma^1_{12}M - \Gamma^2_{12}N.$$
Substituting, this says, ${-\sin(u)(\cos(u)^3 + 1)} = 0$, which is a contradiction.


\section*{Section 4.4}
\subsection*{Question 1}

\begin{enumerate}[(a)]
    \item Let the curve be given an arc-length parameterization
 as $c(s): [0,1] \to C \subseteq S.$
Our goal is to show that $c'''$ is a linear 
combination of $c'$ and $c''$. This would imply that the torsion vanishes, and hence a plane curve.
As $c$ is a geodesic, $c''$ is parallel to the unit normal $\tilde{N}$. 
Moreover as $c$ is a line of curvature, $\frac{\mathrm d\tilde{N}}{\mathrm ds} = -\kappa c$. Combining this gives our result.

\item Conversely, we have that $c'''$ is a linear combination, and we wish to show
that $\frac{\mathrm dc''}{\mathrm ds} = -\kappa c.$ This can be seen by reversing the argument above.

\item Pick our surface as a plane, then all such lines of curvatures are necessarily planar, but not necessarily a geodesic (straight line).

\end{enumerate}

\subsection*{Question 10}
Show that the geodesic curvature of an oriented curve $C \subset S$ at a point $p \in C$ is equal to the curvature of the plane curve obtained by projecting $C$ onto the tangent plane $T_p(S)$ along the normal to the surface at $p$.\\

\textit{Solution.} We parametrize $C$ by $\alpha(t)$ with $\alpha(0)=p$ and $|\alpha'(t)|=1$. Then the projection of 
\begin{equation*}
    \beta(t) = \alpha(t)-\alpha(0) - \langle\alpha(t)-\alpha(0),N\rangle N,
\end{equation*}
where $N$ is the unit normal of $T_p(S)$. Then we have\begin{align*}
    \beta'(t) = \alpha'(t) - \langle\alpha'(t),N\rangle N,
    \beta''(t) = \alpha''(t) - \langle\alpha''(t),N\rangle N.
\end{align*} 
This implies \begin{align*}
    &\beta'(0) = \alpha'(0) - \langle\alpha'(0),N\rangle N = \alpha'(0),\\
    &\beta''(0) = \alpha''(0) - \langle \alpha''(0),N\rangle N;\\
    \Rightarrow  \beta''(0) &= \langle \alpha''(0),R_{90^\circ}\alpha'(0)\rangle R_{90^\circ}\alpha'(0),\\ 
    \Rightarrow  k_\beta &= \langle \beta''(0), R_{90^\circ}\beta'(0)\rangle \\
    &= \langle \beta''(0), R_{90^\circ}\alpha'(0)\rangle\\
    &= \langle \alpha''(0), R_{90^\circ}\alpha'(0)\rangle = k_g.
\end{align*}

\subsection*{Question 20}
Let $T$ be a torus of revolution which we shall assume to be parametrized by \begin{equation*}
    X(u, v)=((r\cos u+a)\cos v,(r\cos u+a)\sin v,r\sin u).
\end{equation*}
Prove that\\

a. If a geodesic is tangent to the parallel $u = \pi/2$, then it is entirely contained in the region of $T$ given by\begin{equation*}
    -\frac{\pi}{2}<u<\frac{\pi}{2}
\end{equation*}

b. A geodesic that intersects the parallel $u = 0$ under an angle $\theta (0< \theta < \pi/2)$ also intersects the parallel $u = \pi$ if \begin{equation*}
    \cos \theta < \frac{a-r}{a+r}.
\end{equation*}\\

\textit{Solution}. 
a. According to Clairaut's relation, given a geodesic $\alpha(t)\subset T$, we have \begin{equation*}
    \frac{d}{dt} R(t)\cos\theta(t) = 0,
\end{equation*}
where $R(t)$ is the distance from $\alpha(t)$ to the $z$-axis and $\theta(t)$ is the angel made by $\alpha'(t)$ and $\mathcal{x}_u$. Suppose $\alpha(0)$ is the tangential point on $u = \pi/2$. Then \begin{equation*}
    R(t)\theta(t) = R(0)\cos\theta(0)= a.
\end{equation*}
In particular, we have $R(t)\geq a$. Hence, $\alpha(t)$ has to lie on the outer side of $T$, which is the region with $-\pi/2<u<\pi/2$.\\

b. Suppose $\alpha(0)$ is the intersection on $u = 0$. We have  \begin{equation*}
    R(t)\theta(t) = R(0)\cos\theta(0) < (a+r)\cdot \frac{a-r}{a+r} = a-r.
\end{equation*}
Suppose $\alpha(t)$ does not intersect $u=\pi$. Let $k = \inf\{u_0\in (-\pi,\pi):\text{$\{u=u_0\}$ intersects $\alpha(t)$}\}$. By the continuity, $\alpha(t)$ will touch $\{u=k\}$ and therefore tangent to the parallel. Moreover, we can also see that $k\neq \pi$. Let the $\alpha(t_0)$ be a tangential point. Then we have \begin{equation*}
    R(t_0)\cos\theta(t_0) = R(t_0) > a - r.
\end{equation*}
This contradicts the fact that $R(t)\cos\theta(t) < a-r$. Hence, 
$\alpha(t)$ has to intersect $\{u=\pi\}$.

\end{document}
