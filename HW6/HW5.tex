\documentclass[12pt]{article}
%% Insert Commands in Preface.tex
\addtolength{\hoffset}{-2.25cm}
\addtolength{\textwidth}{4.5cm}
\addtolength{\voffset}{-2.5cm}
\addtolength{\textheight}{5cm}
\setlength{\parskip}{0pt}
\setlength{\parindent}{15pt}

\usepackage{amsthm, amsmath, amssymb, mathbbol}
\usepackage[colorlinks = true, linkcolor = black, citecolor = black, final]{hyperref}

\newtheorem{theorem}{Theorem}[section]
\newtheorem{corollary}[theorem]{Corollary}
\newtheorem{lemma}[theorem]{Lemma}
\newtheorem{example}[theorem]{Example}
\newtheorem{definition}[theorem]{Definition}

\usepackage{mathrsfs}
\usepackage{graphicx}
\usepackage{multicol}
\usepackage{accents}
\usepackage{tikz}
\usepackage{xcolor}
\usetikzlibrary{patterns}

\setlength{\parindent}{0in}

\pagestyle{empty}

%% Insert Commands Below This Line


\begin{document}

\thispagestyle{empty}

{\scshape Huang Linhang; Syx Pek; Qi Ji} \hfill {\scshape \large Differential Geometry} \hfill {\scshape Homework \#5}

\smallskip
\hrule
\bigskip
\section{Section 4.2}

\subsection*{Question 1}



\subsection*{Question 4}



\subsection*{Question 7}



\subsection*{Question 10}

Let \(S\) be a surface of revolution. Prove that the rotations about its axis are isometries of \(S\).

\emph{Proof.}
Suppose \(S\) is a surface formed by rotating around \(z\)-axis, let \(S\) be parametrised by
\[ \mathbf{x}(u,v) = (\varphi(v)\cos u, \varphi(v)\sin u, \psi(v)) \]
for some \(\varphi, \psi\).
Then the rotation about \(z\)-axis by \(\theta\) can be given by
\[ T = \begin{pmatrix}
    \cos\theta & -\sin\theta & 0 \\
    \sin\theta & \cos\theta & 0 \\
    0 & 0 & 1
\end{pmatrix}. \]
§2-3 exercise 11 shows that \(T\) restricted to \(S\) is a diffeomorphism onto \(S\),
we just need to show that it is a local isometry at every point.

Now let \(\overline{\mathbf{x}} = T\circ\mathbf{x}\) and we can compute that
\[ \overline{\mathbf{x}}(u,v) = (\varphi(v)\cos(u+\theta), \varphi(v)\sin(u+\theta), \psi(v)) \]
Note that §2-3 Example 4 computes the coefficients of the first fundamental form of \(S\) as
\[ E = \varphi^2,\quad F = 0,\quad G = (\varphi')^2 + (\psi')^2. \]
We can compute \(\overline{E},\overline{F},\overline{G}\) by hand as
\begin{align*}
    \overline{\mathbf{x}}_u &= (-\varphi(v)\sin(u+\theta),\varphi(v)\cos(u+\theta), 0) \\
    \overline{\mathbf{x}}_v &= (\varphi'(v)\cos(u+\theta),\varphi'(v)\sin(u+\theta), \psi'(v)) \\
    \overline{E} &= \left\langle \overline{\mathbf{x}}_u, \overline{\mathbf{x}}_u \right\rangle \\
                 &= \varphi^2 = E\\
    \overline{F} &= \left\langle \overline{\mathbf{x}}_u, \overline{\mathbf{x}}_v \right\rangle \\
                 &= 0 = F \\
    \overline{G} &= \left\langle \overline{\mathbf{x}}_v, \overline{\mathbf{x}}_v \right\rangle \\
                 &= (\varphi')^2 + (\psi')^2 = G
\end{align*}
Applying proposition 1 we have \(\overline{\mathbf{x}}\circ\mathbf{x} = T\) is a local isometry at some arbitrary point, which suffices.

\subsection*{Question 13}
Let $V$ and $W$ be (finite-dimensional) vector spaces with inner products $\langle ,\rangle$. Let $G: V \to W$ be a linear map. Prove that the following conditions are equivalent:
\begin{enumerate}
    \item There exists a real constant $\lambda \neq 0$ such that \begin{equation*}
        \langle G(v_1), G(v_2)\rangle = \lambda^2\langle v_1, v_2\rangle \quad \text{for all} v_1,v_2 \in V.
    \end{equation*}
    \item There exists a real constant $\lambda>0$ such that \begin{equation*}
        |G(v)| = \lambda |v|\quad \text{for all}\quad v \in V.
    \end{equation*}
    \item There exists an orthonormal basis $\{v_1,\dots,v_n\}$ of $V$ such that $\{G(v_1),...,G(v_n)\}$ is an orthogonal basis of $W$ and, also, the vectors $G(v_i), i = 1, ...,n$, have the same (nonzero) length.
\end{enumerate}

If any of these conditions is satisfied, G is called a linear conformal map (or a similitude).\\

\textit{Solution.}
($1 \Rightarrow 2$) We have \begin{align*}
    |G(v)| &= \sqrt{\langle G(v), G(v)\rangle}\\
    &= \sqrt{\lambda^2 \langle v, v\rangle}\\
    &= |\lambda|\sqrt{\langle v, v\rangle}\\
    &= |\lambda||v|,
\end{align*}
where $|\lambda| > 0$ is the positive constant desired.\\

($2 \Rightarrow 1$) We have \begin{align*}
    \langle G(v_1), G(v_2)\rangle &= \frac{1}{2}(|G(v_1)+G(v_2)|^2-|G(v_1)|^2-|G(v_2)|^2),\\
    &= \frac{\lambda^2}{2}(|v_1+v_2|^2-|v_1|^2-|v_2|^2)\\
    &= \lambda^2\langle v_1, v_2\rangle\\
\end{align*}

($1\&2 \Rightarrow 3$) For $\{v_i,v_j\}$ orthonormal we have \begin{align*}
    |G(v_i)| = \lambda|v_i| &= \lambda|v_j| = |G(v_j)|\\
    \langle G(v_i), G(v_j)\rangle &= \lambda^2 \langle v_i, v_j\rangle
\end{align*}
This shows that $\{G(v_1),...,G(v_n)\}$ is an orthogonal basis of $W$ and $G(v_i), i = 1, ...,n$, have the same (nonzero) length.\\

($3 \Rightarrow 2$) For any $v \in V$, let $v = \sum_{i=1}^na_iv_i$. Then \begin{align*}
    |G(v)|^2 &= \langle G(v),G(v)\rangle = \langle \sum_{i=1}^n a_iG(v_i),\sum_{i=1}^n a_iG(v_i)\rangle\\
    &=\sum_{i=1}^n\sum_{i=1}^n a_ia_j\langle G(v_i),G(v_j)\rangle\\
    &=\sum_{i=1}^n a_i^2\langle G(v_i),G(v_i)\rangle + \sum_{i\neq j; i,j\in{1,\dots, n}}a_ia_j\langle G(v_i),G(v_j)\rangle\\
    &=\sum_{i=1}^n \lambda^2 a_i^2\langle v_i, v_j\rangle + 0\\
    &=\lambda^2 \langle v,v\rangle = (|\lambda||v|)^2
\end{align*}
Hence, $|G(v)| = |\lambda||v|$.

\subsection*{Question 16}
Let $\mathbf{x}:U \subset R^2 \to R^3$, where \begin{align*}
    U &=\{(\theta,\varphi)\in R^2:0<\theta<\pi,0<\varphi<2\pi\},\\
    x(\theta,\varphi) &= (\sin{\theta}\cos{\varphi},\sin{\theta}\sin{\varphi},\cos{\theta}),
\end{align*}
be a parametrization of the unit sphere $S^2$. Let\begin{equation*}
     \log\tan \frac{1}{2}\theta = u,\qquad \varphi = v
\end{equation*}
and show that a new parametrization of the coordinate neighborhood $\mathbf{x}(U)= V$ can be given by
\begin{equation*}
    \mathbf{y}(u,v) = (\text{sech } u\cos v,\text{sech } u\sin v,\tanh u).
\end{equation*}
Prove that in the parametrization $y$ the coefficients of the first fundamental form are \begin{equation*}
    E=G=\text{sech}^2~u,\qquad F=0.
\end{equation*}
Thus, $\mathbf{y}^{-1} : V \subset S^2 \to R^2$ is a conformal map which takes the meridians and
parallels of $S^2$ into straight lines of the plane. This is called \textit{Mercator's projection}.\\

\textit{Solution.} We have \begin{equation*}
    \theta = 2\arctan e^u, \qquad v = \varphi.
\end{equation*} Hence,\begin{align*}
    \mathbf{y}(u,v) &= \mathbf{x}(2\arctan e^u,v)\\
    &=(\sin{2\arctan e^u}\cos{v},\sin{2\arctan e^u}\sin{v},\cos{2\arctan e^u})\\
    &=(\text{sech }u\cos{v}, \text{sech }u\sin{v},\tanh u).
\end{align*}
Therefore, we have \begin{align*}
    \mathbf{y}_u &= \langle -\tanh u \text{ sech }u\cos v, -\tanh u \text{ sech }u\sin v, 1 - \tanh^2 u \rangle\\
    \mathbf{y}_v &= (-\text{sech }u\sin{v}, \text{sech }u\cos{v},0)\\
    E &= |\mathbf{y}_u|^2 = \tanh^2 u \text{ sech}^2~u + (1-\tanh^2 u)^2\\
    &= \tanh^2 u (1-\tanh^2 u) + (1-\tanh^2 u)^2\\
    &= 1-\tanh^2 u = \text{sech}^2~u\\
    G &= |\mathbf{y}_v|^2 = \text{sech}^2~u (\sin^2 u+\cos^2 u) = \text{sech}^2~u\\
    F &= \tanh u \text{sech}^2~u\sin v\cos v - \tanh u \text{sech}^2~u\sin v\cos v = 0
\end{align*}


\end{document}
