\documentclass[12pt]{article}
%% Insert Commands in Preface.tex
\addtolength{\hoffset}{-2.25cm}
\addtolength{\textwidth}{4.5cm}
\addtolength{\voffset}{-2.5cm}
\addtolength{\textheight}{5cm}
\setlength{\parskip}{0pt}
\setlength{\parindent}{15pt}

\usepackage{amsthm, amsmath, amssymb, mathbbol}
\usepackage[colorlinks = true, linkcolor = black, citecolor = black, final]{hyperref}

\newtheorem{theorem}{Theorem}[section]
\newtheorem{corollary}[theorem]{Corollary}
\newtheorem{lemma}[theorem]{Lemma}
\newtheorem{example}[theorem]{Example}
\newtheorem{definition}[theorem]{Definition}

\usepackage{mathrsfs}
\usepackage{graphicx}
\usepackage{multicol}
\usepackage{accents}
\usepackage{tikz}
\usepackage{xcolor}
\usetikzlibrary{patterns}

\setlength{\parindent}{0in}

\pagestyle{empty}

%% Insert Commands Below This Line


\begin{document}

\thispagestyle{empty}

{\scshape Huang Linhang; Syx Pek; Qi Ji} \hfill {\scshape \large Differential Geometry} \hfill {\scshape Homework \#2}

\smallskip
\hrule
\bigskip

\section{Section 2.3}

\subsection*{Question 1}
Let $S^2 = \{(x,y,z) \in \mathbb{R}^3:x^2+y^2+z^2= 1\}$ be the unit sphere and let $A:S^2 \to S^2$ be the (antipodal) map $A(x,y, z) = (-x, -y, -z)$. Prove that $A$ is a diffeomorphism.\\

\textit{Solution.}
Let $U = \{(x,y,\sqrt{1-x^2-y^2}):x^2+y^2<1\}$. Then we have \begin{align*}
    A(U) &=\{-(x,y,\sqrt{1-x^2-y^2}):x^2+y^2<1\}\\
    &=\{(x,y,-\sqrt{1-x^2-y^2}):x^2+y^2<1\}.
\end{align*}

Let $\phi(u,v) = (u,v,\sqrt{1-u^2-v^2})$ be a parametrization for $U$ and $\psi(u,v) = (u,v,-\sqrt{1-u^2-v^2})$ a parametrization for $A(U)$. Then \begin{align*}
    \psi^{-1}\circ A\circ\phi(u,v) &= \psi^{-1}(A(\phi(u,v)))\\
    &= \psi^{-1}(A((u,v,\sqrt{1-u^2-v^2})))\\
    &= \psi^{-1}((-u,-v,-\sqrt{1-u^2-v^2}))\\
    &= -(u,v).
\end{align*}

Therefore, $\psi^{-1}\circ A\circ\phi$ is a differentiable function from $U\to A(U)$. This shows that $A$ is a diffeomorphism.

\subsection*{Question 3}

Show that the paraboloid $z = x^2 + y^2$
is diffeomorphic to the plane.

\textit{Solution.}  Luckily, we can pick one local chart each
to cover the paraboloid $U$ and the plane $V$. For the paraboloid, we can have $(U, f^{-1})$ where
$f: (x, y) \to (x, y, x^2 + y^2)$ and for the plane, we can choose our local chart to be $(V, g^{-1})$ where
to be $g: (x, y) \to (x, y).$

Let us now define the diffeomorphism 
$\pi : U \to V.$
Let $\pi (a, b, c) = (a, b).$ Then, $g^{-1}\circ\pi \circ f = \mathrm{id}_{\mathbb R^2}.$ Thus, $\pi$ is smooth.
Then, $\pi^{-1}$ is well-defined as it is bijective as a function of sets.
$\pi^{-1}$ is smooth as $f^{-1} \circ \pi^{-1} \circ g = \mathrm{Id}_{\mathbb R^2}.$ Thus, $\pi$ is a diffeomorphism.

\subsection*{Question 5}
Let $S \subset R^3$ be a regular surface, and let $d:S\to R$ be given by $d(p)=|p- p_0|$, where $p \in S, p \in R^3, p_0 \notin S$; that is, $d$ is the distance from $p$ to a fixed point $p_0$, not in $S$. Prove that $d$ is differentiable.\\

\textit{Solution.} We select $p\in S$, a open neighborhood $U$ of $p$ and a parametrization $\utilde{x}: V \to U$, where $V\subset R^2$ and $\utilde{x}(u_0,v_0) = p$. Let $p_0=(p_1,p_2,p_3)$. Let $f(p) = (d(p,p_0))^2$Then \begin{align*}
    f\circ\utilde{x}(u,v) &= (x_1(u,v)-p_1)^2+(x_2(u,v)-p_2)^2+(x_3(u,v)-p_3)^2
\end{align*}
Since $x_i(u,v)$ are all smooth functions, $f\circ \utilde{x}$ should also be smooth. Since $p_0\notin S$, we have $f\circ\utilde{x}(S)\in (0,\infty)$. Note that $g(x)=\sqrt{x}$ is smooth on $(0,\infty)$. Therefore, $\sqrt{f\circ\utilde{x}} = d\circ\utilde{x}$ is also smooth. This shows that $d$ is differentiable.

\subsection*{Question 7}

\begin{definition}
    A \textbf{manifold} is a second-countable Hausdorff topological space $M$ such that
    for any point $m \in M$, there exists an open set $N$ containing $m$ such that $N$ is homeomorphic to a subspace of $\mathbb R^k$ for some fixed $k \in \mathbb Z_{\geq 0}$.

    A \textbf{chart} is a pair $(U, \phi)$ such that $U$ is an open subset of $M$ and $\phi:U \to \mathbb R^k$
    is a homeomorphism. Then, an \textbf{atlas} is a set of charts such that for every point, there exists a chart
    $(A, \alpha)$ such that $A$ contains the point.

    A \textbf{differentiable manifold} is a manifold with an atlas with the property that
    for any two charts $(A, \alpha)$ and $(B, \beta)$ in the atlas, if $A\cap B$ is nonempty, $\beta\circ \alpha^{-1}$
    is a differentiable map when restricted such that the image of $\alpha^{-1}$ is $A\cap B$.
\end{definition}


\begin{remark}
    A surface is a manifold, as it is by definition Hausdorff and second countable as a subspace of $\mathbb R^3$,
    and locally is isomorphic to $\mathbb R^2$ as per the definiton of surface.
    A regular surface implies that we have a atlas with compatible differentiable structure.
\end{remark}


\begin{theorem} Diffeomorphism is an isomorphism in the category of Differentiable Manifolds.
\end{theorem}

\begin{proof}
    We need to show reflexivity, symmetric and transititive.
    Let $(M, \mathscr A)$ and $(N, \mathscr B)$ be two manifolds where $\mathscr A, \mathscr B$ are
    maximal atlas. Let $f: M \to N$ be a diffeomorphism.
    \begin{itemize}
        \item (Reflexive) The diffeomorphism has the property
        that $x \in \mathscr B$ if and only if $x \circ f \in \mathscr A$,
        and $f$ is bijective (as sets). Thus, $y \in \mathscr A$ implies, $y \circ f^{-1} \circ f \in \mathscr A$,
        implying $y \circ f^{-1} \in \mathscr B$. As the implication are reversible,
        this shows that $f^{-1}$ is also a diffeomorphism.
        \item (Symmetric) Consider the identity map on $M$. This has the property
        that $x \in \mathscr A$ if and only if $x \circ \mathrm{Id}_M \in A$.
        \item (Transitivity) Let $(K, \mathscr C)$ be a manifold where $\mathscr C$ is a maximal atlas,
        and $g$ be a diffeomorphism from $N$ to $K$.
        Then, $x \in \mathscr C$ if and only if $x \circ g \in \mathscr B$ if and only if $x \circ g \circ f \in \mathscr A.$
        As the composition of bijective functions is bijective, $(g \circ f)$ is a diffeomorphism.
    \end{itemize} 
\end{proof}

\subsection*{Question 9}

\textbf{(a).} Define the notion of differentiable function on a regular curve.
              What does one need to prove for the definition to make sense?

\emph{Solution.}
Let \(f: C\to \mathbb{R}\) be defined on a regular curve \(S\).
Then \(f\) is differentiable at \(p\in C\) if,
for some parametrization \(\alpha: I\to C\) with \(p\in \alpha(I)\subset C\) where \(I\) is an interval,
the composition \(f\circ \alpha: I\to \mathbb{R}\) is differentiable at \(\alpha^{-1}(p)\).
\(f\) is differentiable in \(C\) if it is differentiable at all points \(p\in C\).

For the definition to make sense, one has to prove that the definition given does not depend on the choice of parametrization.
One needs to prove a statement analogous to proposition 1, that is given two parametrizations of \(C\) both containing a neighbourhood of \(p\),
the change of parameters between those is a diffeomorphism.

\section{Section 2.4}

\subsection*{Question 15}
Show that if all normals to a connected surface pass through a fixed point, the surface is contained in a sphere.\\

\textit{Solution.} Let $p_0$ be such a fixed point. Then for any $p = \utilde{x}(u,v)\in S$, $p-p_0$ is normal to the tangent plane in $p$. That is \begin{equation*}
    \utilde{x}_u\cdot (p-p_0) = \utilde{x}_v\cdot (p-p_0) = 0.
\end{equation*}
Then for the function $h(u,v)=(\utilde{x}(u,v)-p_0)\cdot(\utilde{x}(u,v)-p_0),$ we have \begin{align*}
    &h_u = 2\utilde{x}_u\cdot (\utilde{x}-p_0) = 0\\
    &h_v = 2\utilde{x}_v\cdot (\utilde{x}-p_0) = 0.
\end{align*}
This shows that $h$ is constant for any connected component of $S$. However, since $S$ is connected, $h$ should be constant on $S$, which implies that $S\subset {p\in R^3:||p-p_0|| = K}$ for some $K>0$.

\subsection*{Question 17}

Two regular surfaces $S_1$ and $S_2$ intersect transversally if whenever $p \in S_1 \cap S_2$,
then $T_p(S_1) \neq T_p(S_2)$. Prove that if $S_1$ intersects $S_2$ transversally,
then a connected component of $S_1 \cap S_2$ is a regular curve.

\begin{proof}
    Observe in $\mathbb R^3$, if $T_p(S_1) \neq T_p(S_2)$,
    then $\dim_\mathbb{R} (T_p(S_1) \cap T_p(S_2)) = 1.$

    It suffices to show that $S_1 \cap S_2$ is a locally a curve.
    This can be seen as we pick $v \in T_p(S_1)\cap T_p(S_2)$. Then
    we have a curve $c: [0,1] \to S_1$ with initial conditions $c(0) = p$,$c'(0) = v$.
    Similarly, we have a curve $d:[0,1] \to S_2$ mutatis mutandis.
    As they share initial conditions, they are locally equal at $p$, and thus the intersection is locally a curve.
\end{proof}

\subsection*{Question 19}

Let \(S \in \mathbb{R}^3\) be a regular surface and \(P \in \mathbb{R}^3\) be a plane.
If all points of \(S\) are on the same side of \(P\), prove that \(P\) is tangent to \(S\) at all points of \(P \cap S\).

\emph{Proof.}
Let \(p \in P\) and \(n_P\) be a unit normal of the plane \(P\), then consider \[ h(r) = (r - p)\cdot n_P \]
one can compute that \(dh_r = n_P\) for all \(r\in \mathbb{R}^3\).

Without loss of generality assume that for all \(s\in S\), \(h(s) \geq 0\), so the surface lies on the ``positive'' side of the plane.
If \(\gamma(u)\) is a regular path in \(S\), then
\[ (h\circ \gamma)'(u) = (\nabla h)\cdot \gamma'(u) = n_P \cdot \gamma'(u).\]
Now if \(s \in P\cap S\), then \(h(s) = 0\),
and if \(\gamma\) is a path that goes through \(s\) such that \(\gamma(u_0) = s\),
then we know that \((h\circ\gamma)(u) \geq (h\circ\gamma)(u_0)\) and by extreme value theorem we have
\[ (h\circ\gamma)'(u_0) = 0 = \gamma'(u_0)\cdot n_P. \]
Since the path \(\gamma\) chosen was arbitrary, it means that the entire tangent plane of \(s\) is normal to \(n_P\) at \(s\), which means that its tangent plane is \(P\).

\subsection*{Question 21}

Let \(f: S\to R\) be a differentiable function on a connected regular surface \(S\).
Assume that \(df_p = 0\) for all \(p\in S\).
Prove that \(f\) is constant on \(S\).

\emph{Proof.}
Suppose for a contradiction that \(f\) is non-constant, let \(f(s_1) = y_1\) and \(f(s_2) = y_2\) where \(s_1 \ne s_2\) and \(y_1 \ne y_2\).
Since \(S\) is connected, consider a regular path \(\gamma\) from \(s_1\) to \(s_2\), let \(\gamma(a) = s_1\) and \(\gamma(b) = s_2\).
By chain rule
\[ (f\circ \gamma)'(x) = df_{\gamma(x)}\cdot \gamma'(x) = 0 \]
from single variable calculus we have \(f\circ\gamma\) is a constant function,
but \((f\circ\gamma)(s_1) = y_1 \ne y_2 = (f\circ\gamma)(s_2)\).

\subsection*{Question 23}

Let $U = \{(x, y, z) \in \mathbb{R}^3\ |\ z = -1\}$ be identified
with the complex plane $\mathbb C$, by sending $(x, y, z) \in U$ to $x + iy.$
Let $P(x) = \sum_{k = 0}^n a_k x^{n-k} \in \mathbb C[x]$ be of degree $n$.
Denote $\pi$ to be the stereographic projection
of $S^2 = \{(x, y, z) \in \mathbb{R}^3\ |\ x^2 + y^2 + z^2 = 1\}$
from the north pole $(0,0,1)$ onto $U$.

Prove that the map $F: S^2 \to S^2$ given by
\begin{align*}
    F(p) &= \pi^{-1}\circ P \circ \pi(p)\text{ for }p \not\in U\setminus\{(0,0,1)\}
    \\F((0,0,1)) &= (0,0,1)
\end{align*}
has finitely many critical points.

\begin{proof}
    Recall that $\pi(x, y, z) = \left(\frac{2x}{1-z}, \frac{2y}{1-z}, -1\right) \subseteq U$.
    Thus $\pi$ is a diffeomorphism between $U\setminus\{(0,0,1)\}$ and $\mathbb R^2$. Thus, it suffices to
    show that $\hat{P} : \mathbb R^2 \to \mathbb R^2$ sending $(x, y)$ to $(\mathrm{Re} P(x+iy), \mathrm{Im} P(x+iy))$
    has finitely many critical points. (We can ignore the north pole, as it doesn't contain infinitely many distinct points).
    To see this, we require $\frac{\partial \mathrm{Re} P(x+iy)}{\partial x}=\frac{\partial \mathrm{Re} P(x+iy)}{\partial y}=\frac{\partial \mathrm{Im} P(x+iy)}{\partial x}=\frac{\partial \mathrm{Im} P(x+iy)}{\partial y} = 0.$
    By Complex Analysis, this implies that $\frac{\mathrm dP(z)}{dz} = 0$, but that can only occurs finitely many times as $P$ is a polynomial.
\end{proof}

\subsection*{Question 25}
Prove that if two regular curves $C_1$ and $C_2$ of a regular surface $S$ are tangent at a point $p \in S$, and if $\varphi: S \to S$ is a diffeomorphism, then $\varphi(C_1)$ and $\varphi(C_2)$ are regular curves which are tangent at $\varphi(p)$.\\

\textit{Solution.} Let $U$ be a neighborhood of $p$ with parametrization $\utilde{x}(u,v)$. Let $\alpha_1(t)$ and $\alpha_2(t)$ be such that $\utilde{x}\circ \alpha_1$ and $\utilde{x}\circ \alpha_2$ are regular parametrizations of $C_1$ and $C_2$ with $\utilde{x}\circ \alpha_1(0) = \utilde{x}\circ \alpha_2(0)=p$. Then that $C_1$ and $C_2$ are tangent at $p$ implies that \begin{equation*}
    \alpha_1'(0) = \alpha_2'(0).
\end{equation*}
Now let $V$ be a neighborhood of $\varphi(p)$ with parametrization $\utilde{y}(w,z)$ and $\psi =(\psi_1,\psi_2)= \utilde{y}^{-1}\circ \varphi\circ \utilde{x}$. Let $\beta_1(t)$ and $\beta_2(t)$ be such that $\utilde{y}\circ \beta_1$ and $\utilde{y}\circ \beta_2$ are regular parametrizations of $\varphi(C_1)$ and $\varphi(C_2)$ with $\utilde{y}\circ \beta_1(0) = \utilde{y}\circ \beta_2(0)=\varphi(p)$. Than
\begin{align*}
    \beta_1'(0) &= \left(\frac{\partial \psi_1}{\partial u}(\alpha_{1}(0))\alpha_{1u}'(0)+\frac{\partial \psi_1}{\partial v}(\alpha_{1}(0))\alpha_{1v}'(0),\frac{\partial \psi_2}{\partial u}(\alpha_{1}(0))\alpha_{1u}'(0)+\frac{\partial \psi_2}{\partial v}(\alpha_{1}(0))\alpha_{1v}'(0)\right)\\
    &=\begin{pmatrix}
    \frac{\partial \psi_1}{\partial u}(\alpha_{1}(0))&\frac{\partial \psi_1}{\partial v}(\alpha_{1}(0))\\
    \frac{\partial \psi_2}{\partial u}(\alpha_{1}(0))&\frac{\partial \psi_2}{\partial v}(\alpha_{1}(0))
    \end{pmatrix}\alpha_1'(0)\\
    &= J_\psi(\alpha_1(0))\alpha_1'(0).
\end{align*}
Likewise, we have \begin{equation*}
    \beta_2'(0) = J_\psi(\alpha_2(0))\alpha_2'(0).
\end{equation*}
However, since $\alpha_1(0)=\alpha_2(0) = \utilde{x}^{-1}(p)$ and $\alpha'_1(0)=\alpha'_2(0)$, we have $\beta'_1(0)=\beta'_2(0)$. This implies that $\varphi(C_1)$ and $\varphi(C_2)$ is tangent at $\varphi(p)$.
\end{document}
