\documentclass[12pt]{article}
%% Insert Commands in Preface.tex
\addtolength{\hoffset}{-2.25cm}
\addtolength{\textwidth}{4.5cm}
\addtolength{\voffset}{-2.5cm}
\addtolength{\textheight}{5cm}
\setlength{\parskip}{0pt}
\setlength{\parindent}{15pt}

\usepackage{amsthm, amsmath, amssymb, mathbbol}
\usepackage[colorlinks = true, linkcolor = black, citecolor = black, final]{hyperref}

\newtheorem{theorem}{Theorem}[section]
\newtheorem{corollary}[theorem]{Corollary}
\newtheorem{lemma}[theorem]{Lemma}
\newtheorem{example}[theorem]{Example}
\newtheorem{definition}[theorem]{Definition}

\usepackage{mathrsfs}
\usepackage{graphicx}
\usepackage{multicol}
\usepackage{accents}
\usepackage{tikz}
\usepackage{xcolor}
\usetikzlibrary{patterns}

\setlength{\parindent}{0in}

\pagestyle{empty}

%% Insert Commands Below This Line


\begin{document}

\thispagestyle{empty}

{\scshape Huang Linhang; Syx Pek; Qi Ji} \hfill {\scshape \large Differential Geometry} \hfill {\scshape Homework \#2}

\smallskip
\hrule
\bigskip

\section{Section 1.7}

\subsection*{Question 6}

Let \(\alpha(s), s\in[0,l]\) be a closed convex plane curve positively oriented. The curve \[\beta(s) = \alpha(s) - rn(s)\]
where \(r\) is a positive constant and \(n\) is a normal vector, is called a \emph{parallel} curve to \(\alpha\) (textbook figure is wrong, curve needs to be convex).
Show that

(a) \(\text{Length of }\beta = \text{length of }\alpha + 2\pi r\).

\textit{Solution.}
WLOG assume \(\alpha\) is parametrised by arc length, since \(\alpha\) is a plane curve the torsion is zero, so \(n' = -kt\).
\begin{align*}
    \beta'(s) &= \alpha'(s) - r n'(s) \\
              &= \alpha'(s) - r(-k(s) t(s)) \\
              &= (1+rk(s))t(s)
\end{align*}
now as \(\alpha\) is a simple closed curve with positive orientation, the rotation index is \(1\) and \(\int_0^l k(s)\ ds = 2\pi\), so
\begin{align*}
    \text{length of }\beta
    &= \int_0^l \left\lvert \beta'(s) \right\rvert\ ds \\
    &= \int_0^l 1 + r k(s) \ ds \\
    &= l + 2\pi r.
\end{align*}

(b) \(A(\beta) = A(\alpha) + rl + \pi r^2\).

\textit{Solution.}
For each \(t\in[0,r]\) let
\[\beta_t(s) = \alpha(s) - t n(s)\]
which is a curve that is also parallel to \(\alpha\) and lies between \(\alpha\) and \(\beta\).
Then we have the expression
\begin{align*}
    A(\beta) - A(\alpha) &= \int_0^r \text{length of }\beta_t \ dt \\
                         &= \int_0^r l + 2\pi t\ dt \\
                         &= rl + \pi r^2
\end{align*}
as required.

(c) \(k_\beta(s) = k_\alpha(s)/(1+r k_\alpha(s))\) (question typo'd).

\textit{Solution.}
We re-parametrise \(\beta\) by arc length.
Let \[t(s) = \int_0^s \lvert \beta'(u) \rvert\ du\]
and we have \(\frac{ds}{dt} = \frac{1}{\lvert \beta'(s)\rvert}\).
Define \(\hat{\beta}\) such that \(\hat{\beta}(t(s)) = \beta(s)\), then
\[\hat{\beta}''(l) = \frac{\alpha''(s)}{\lvert \beta'(s) \rvert} \]
taking lengths we have
\[\lvert \hat{\beta}''(l) \rvert = k_\beta(s) = \frac{k_\alpha(s)}{1 + rk_\alpha(s)}. \]

\subsection*{Question 7}

Let $\alpha : \mathbb R \to \mathbb R^2$ be a plane curve
defined in the entire real line $\mathbb R$.
Assume that $\alpha$ does not pass through
the origin $O = (0,0)$ and that both limits
$$\lim_{t\to -\infty} |\alpha(t)| = \lim_{t\to \infty} |\alpha(t)| = \infty.$$

(a) Prove that there exists a point $t_0$ such that $|\alpha(t_0)| \leq |\alpha(t)|$ for all $t \in \mathbb R.$

(b) Show, by an example, that the assertion in part a is false
if one does not assume that both $\lim_{t \to -\infty}|\alpha(t)| = \infty$ and $\lim_{t \to \infty}|\alpha(t)| = \infty$.

\textit{Solution.}

(a) Consider $f : \mathbb{R} \to \mathbb{R}$ sending $t$ to $|\alpha(t)|.$
Pick an arbitrary $t_1 \in \mathbb{R}$. We can find $a, b$ such that for all $t < a$,
$f(t) > f(t_1)$ and for all $t > b$, we have $f(t) > f(t_1).$ Then, we can consider
the restriction $f|_{[a,b]}$ to the compact interval $[a,b]$. By the Extreme Value Theorem,
we have that this takes a minimum on $[a, b]$, say at $t_0$. Then this $t_0$ satisfy the required properties
as for $t \in [a,b]$, $|\alpha(t_0)| \leq |\alpha(t)|$, and for $t \not \in [a,b]$, $|\alpha(t)| > |\alpha(t_1)| \geq |\alpha(t_0)|$.

(b) Consider $f(t) = (e^t, 0).$ Then, $\inf_t |\alpha(t)| = 0$, but this is non-zero for all $t$.

\subsection*{Question 8}

(a) Let \(\alpha(s)\), \(s\in[0,l]\), be a plane simple closed curve.
Assume that the curvature \(k(s)\) satisfies \(0<k(s)\leq c\) where \(c\) is a constant
(thus, \(\alpha\) is less curved than a circle of radius \(1/c\)). Prove that
\[ \text{length of }\alpha \geq \frac{2\pi}{c} . \]

\textit{Solution.}
We have a simple closed curve so we know that
\[ \int_0^l k(s)\ ds = 2\pi \]
substituting the bounds \(0<k(s)\leq c\) we have
\[ 2\pi \leq cl \implies l \geq \frac{2\pi}c. \]

(b) In part (a) replace the assumption of being simple by ``\(\alpha\) has rotation index \(N\).'' Prove that
\[ \text{length of }\alpha \geq \frac{2\pi N}c. \]

\textit{Solution.}
Similarly, as rotation index is \(N\),
\[ \int_0^l k(s)\ ds = 2\pi N \]
similarly substitute the bounds and we have
\[ l \geq \frac{2\pi N}c. \]

\subsection*{Question 9}

We here define the notion of being ``inside'' a simple closed curve.

\begin{definition}
    A point $p$ is inside a simple closed curve $C$ if there
    exists a ray from $p$ that intersects exactly one point in $C$, say $q$,
    such that $q \neq p$. Moreover, the tangent line at $q$
    is not parallel to the ray from $p$ to $q$.
\end{definition}

\begin{theorem}
    A set $K \subseteq \mathbb R^2$ is convex
    if for any two points $p, q \in K$, the segment of 
    straight line $\overline{pq}$ is contained in $K$. 

    Let $C$ be a simple closed curve and $K$ be the points on or inside $C$.
    Then if $C$ is convex so is $K$.
\end{theorem}

\begin{proof}
    For each point $c \in C$, let $U_c$ be the closed half-plane, 
    with the tangent line through $c$ being the boundary and
    $C \subseteq U_c$. By convexity, we have $K \subseteq U = \cap_{c \in C} U_c.$
    As the intersection of two convex sets is convex, and half-planes are convex,
    it remains to show that $U \subseteq K$.

    Consider a point $p \in \mathbb{R}^2 \setminus K$.
    As $C$ is compact, we can pick $q \in C$ such that $d(p, q)$ is minimized.
    Then, the tangent line at $q$ must be perpendicular to the ray from $p$ to $q$.
    Suppose then that $p \in U_q$. Then the ray can not intersect another point of $C$,
    as $C$ cannot intersect the line segment $\overline{pq}$ as $q$ is the closest point to $p$ on $C$,
    and $C$ cannot intersect the ray after $q$, as all of $C$ lies in $U_q.$
    By our definition, this implies that $p$ is inside the curve, which is a contradiction.
\end{proof}

\section{Section 2.2}

\subsection*{Question 4}
Let $f(x, y, z) = z^2$. Prove that $0$ is not a regular value off and yet that $f^{-1}(0)$ is a regular surface.\\

\textit{Solution.} Note that \begin{equation*}
    \mathrm df = (f_x,f_y,f_z) = (0,0,2z),
\end{equation*}
which is not surjective only when $z=0$. Hence, $(0,0,0)$ is a critical point and thus $f(0,0,0)=0$ is not a regular value.\\

However, \begin{align*}
    f^{-1}(0) &= \{(x,y,z)\in \mathbb{R}^3|z^2 = 0\}\\
    &= \{(x,y,0)|x,y \in \mathbb{R}\} \\
    &= \mathbb{R}^2 \times \{0\}.
\end{align*}
Hence, $f^{-1}(0)$ is homeomorphic to $\mathbb{R}^2$ and therefore is regular.

\subsection*{Question 5}
Let $P= \{(x,y,z)\in \mathbb{R}^3|x= y\}$ (a plane) and let $x:U \subset \mathbb{R}^2 \to \mathbb{R}^3$ be given by \begin{equation*}
    x(u,v) = (u+v,u+v,uv),
\end{equation*}
 where $U=\{(u,v)\in \mathbb{R}^2|u>v\}$. Clearly, $x(U)\subset P$.Is $x$ a parametrization of $P$?\\

\textit{Solution.} Yes, $x$ is a parametrization. Clearly, $x$ is differentiable in $U$ with \begin{equation*}
    \mathrm dx(u,v) =\begin{pmatrix}
    1&1\\
    1&1\\
    u&v\\
    \end{pmatrix}.
\end{equation*}
Note that for $(u,v)\in U$, we have $u>v$. Then $$\begin{array}{|cc|}
     1&1  \\
     u&v
\end{array} = v - u \neq 0.$$
This implies $\mathrm dx(u,v)$ is injective for all $(u,v)\in U$. Now let $(a,a,b)$ be any point in $x(U)$. Then \begin{align*}
    &u+v=a,uv=b\\
    \Rightarrow &u(a-u)=b\\
    \Rightarrow &(u-\frac{a}{2})^2 = \frac{a^2}{4}-b.
\end{align*}
Notice that here one must have $\frac{a^2}{4}-b \geq 0$ as the equations $\begin{cases} u+v=a\\
uv=b
\end{cases}$ should have real solutions for $(a,a,b)\in x(U)$. Then given $u>v$, we have \begin{align*}
    u = \frac{a}{2}+\sqrt{\frac{a^2}{4}-b}\\
    v = \frac{a}{2}-\sqrt{\frac{a^2}{4}-b}.
\end{align*}

These are the unique $(u,v)$ solving $x(u,v)=(a,b)$, which shows $x$ is injective. Hence, by Prop. 4,  $x^{-1}$ must be continuous and we can conclude that $x$ is indeed a parametrization.

\subsection*{Question 6}
Give another proof of Prop. 1 by applying Prop. 2 to $h(x,y,z)=f(x,y)-z$.\\

\textit{Solution.}
Since $f$ is differentiable in $U$, for any point in $U \times \mathbb{R}$, we have \begin{equation*}
    \mathrm dh = (f_x,f_y,-1),
\end{equation*}
which is always surjective regardless of the value of $f_x,f_y$. Hence, any $z_0 \in f(U)$ with $f(x_0,y_0) = z_0$, we have \begin{equation*}
    h(x_0,y_0,z_0) = f(x_0,y_0) - z_0 = 0,
\end{equation*}
being a regular value. This implies that \begin{align*}
    h^{-1}(0) &= \{(x,y,z)\in U\times \mathbb{R}|h(x,y,z)=0\}\\
    &=\{(x,y,z)\in U\times \mathbb{R}|f(x,y)=z\}\\
    &=\{(x,y,f(x,y))|(x,y)\in U\}
\end{align*}
is a regular surface.
\end{document}
