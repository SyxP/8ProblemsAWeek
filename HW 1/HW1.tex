\documentclass[12pt]{article}
%% Insert Commands in Preface.tex
\addtolength{\hoffset}{-2.25cm}
\addtolength{\textwidth}{4.5cm}
\addtolength{\voffset}{-2.5cm}
\addtolength{\textheight}{5cm}
\setlength{\parskip}{0pt}
\setlength{\parindent}{15pt}

\usepackage{amsthm, amsmath, amssymb, mathbbol}
\usepackage[colorlinks = true, linkcolor = black, citecolor = black, final]{hyperref}

\newtheorem{theorem}{Theorem}[section]
\newtheorem{corollary}[theorem]{Corollary}
\newtheorem{lemma}[theorem]{Lemma}
\newtheorem{example}[theorem]{Example}
\newtheorem{definition}[theorem]{Definition}

\usepackage{mathrsfs}
\usepackage{graphicx}
\usepackage{multicol}
\usepackage{accents}
\usepackage{tikz}
\usepackage{xcolor}
\usetikzlibrary{patterns}

\setlength{\parindent}{0in}

\pagestyle{empty}

%% Insert Commands Below This Line


\begin{document}

\thispagestyle{empty}

{\scshape Huang Linhang; Syx Pek; Qi Ji} \hfill {\scshape \large Differential Geometry} \hfill {\scshape Homework \#1}
 
\smallskip
\hrule
\bigskip

\section{Section 1.3}

\subsection*{Question 2}

\subsection*{Question 3}

\subsection*{Question 4}

\subsection*{Question 5}

\subsection*{Question 10}


\section{Section 1.4}

\subsection*{Question 10}
The natural orientation of $R^2$ makes it possible to associate a sign to the area $A$ of a parallelogram generated by two linearly independent vectors $u,v \in R^2$. To do this, let $\{e_i\}, i = 1, 2$, be the natural ordered basis of $R^2$, and write $u = u_1e_1 +u_2e_2, v = v_1e_1 +v_2e_2$. Observe the matrix relation \begin{equation*}
    \begin{pmatrix}
    u \cdot u & u \cdot v\\
    v \cdot u & v \cdot v
    \end{pmatrix} = 
    \begin{pmatrix}
    u_1 & u_2\\
    v_1 & v_2
    \end{pmatrix}
    \begin{pmatrix}
    u_1 & v_1\\
    u_2 & v_2
    \end{pmatrix}
\end{equation*}
and conclude that \begin{equation*}
    A^2 = \begin{array}{|cc|}
         u_1 & u_2\\
         v_1 & v_2
    \end{array}^2.
\end{equation*}
Since the last determinant has the same sign as the basis $\{u,v\}$, we can say that $A$ is positive or negative according to whether the orientation of $\{u,v\}$ is positive or negative. This is called the \textit{oriented area} in $R^2$.\\\\
Solution:
\begin{align*}
    A^2 &= (u \wedge v) \cdot (u \wedge v)\\
    &= u \cdot (v \wedge (u \wedge v))\\
    &= u \cdot [(v \cdot v)u - (v \cdot u)v]\\
    &= (u \cdot u)(v \cdot v) - (v \cdot u)(v \cdot u)\\
    &= \begin{array}{|cc|}
    u \cdot u & u \cdot v\\
    v \cdot u & v \cdot v
    \end{array}\\
    &= \begin{array}{|cc|}
    u_1 & u_2\\
    v_1 & v_2
    \end{array}~
    \begin{array}{|cc|}
    u_1 & v_1\\
    u_2 & v_2
    \end{array}\\
    &=\begin{array}{|cc|}
    u_1 & u_2\\
    v_1 & v_2
    \end{array}^2.
\end{align*}
\subsection*{Question 11}
a. Show that the volume $V$ of a parallelepiped generated by three linearly independent vectors $u,v,w \in R^3$ is given by $V=|(u \wedge v)\cdot w|$, and introduce an oriented volume in $R^3$.\\
b. Prove that \begin{equation*}
     V^2 = \begin{array}{|ccc|}
     u \cdot u & u \cdot v & u \cdot w\\
     v \cdot u & v \cdot v & v \cdot w\\
     w \cdot u & w \cdot v & w \cdot w
     \end{array}
 \end{equation*}\\\\
Solution.
(a) Let $n = \frac{u \wedge v}{||u \wedge v||}$ be the normal vector of the plane generated by $u$ and $v$. Then \begin{align*}
     V &= (||u||\times||v||\times|\sin(u,v)|)\times||w||\times|\cos(n,w)|\\
     &= ||u\wedge v||\times||w||\times|\cos(n,w)|\\
     &= ||u\wedge v||\times||w||\times|\cos(u\times v,w)|\\
     &= |(u\wedge v)\cdot w|
 \end{align*}
(b) We know that \begin{align*}
     |(u\wedge v)\cdot w| &= |\begin{pmatrix}
     u_2v_3-v_2u_3\\
     u_3v_1-v_3u_1\\
     u_1v_2-v_1u_2
     \end{pmatrix}\cdot w|\\
     &=\begin{array}{|ccc|}
     u_1 & u_2 & u_3\\
     v_1 & v_2 & v_3\\
     w_1 & w_2 & w_3
     \end{array}\\
     &=\det (u,v,w).
 \end{align*}
Hence, \begin{align*}
     V^2 &= \det (u,v,w)^2\\
     &=|\begin{pmatrix}
     u^T\\
     v^T\\
     w^T
     \end{pmatrix}
     \begin{pmatrix}
      u&v&w
     \end{pmatrix}|\\
     &=\begin{array}{|ccc|}
     u^T u & u^T v & u^T w\\
     v^T u & v^T v & v^T w\\
     w^T u & w^T v & w^T w
     \end{array}\\
     &=\begin{array}{|ccc|}
     u \cdot u & u \cdot v & u \cdot w\\
     v \cdot u & v \cdot v & v \cdot w\\
     w \cdot u & w \cdot v & w \cdot w.
     \end{array}
 \end{align*}
\subsection*{Question 12}
Given the vectors $v \neq 0$ and $w$, show that there exists a vector $u$ such that $u \wedge v = w$ if and only if $v$ is perpendicular to $w$. Is this vector $u$ uniquely determined? If not, what is the most general solution?\\\\
Solution: $(\Rightarrow)$ By the properties of cross product, $u \wedge v = w$ implies that $v \cdot w = 0$.
 
$(\Leftarrow)$ If $v\cdot w=0$, we have \begin{align*}
     (v\wedge w)\wedge v &= (v\cdot v)w - (v\cdot w)v=||v||^2w.
\end{align*}
Then $v\neq 0$ implies that $w = \frac{v\wedge w}{||v||^2}\wedge v$. Let $u=\frac{v\wedge w}{||v||^2}$ and we have $u\wedge v = w$.
 
Suppose there exist $u'$ other than $u$ such that $u'\times v = w$. Then \begin{align*}
     &u \wedge v = u' \wedge v \\
     \Rightarrow &(u'-u)\wedge v = 0\\
     \Rightarrow &u'-u = kv, \quad k\in R\\
     \Rightarrow &u' = u + kv, \quad k\in R.
\end{align*}
 
Therefore, the most general solution of $u\wedge v=w$ is \begin{equation*}
     u = \frac{v\wedge w}{||v||^2} + kv,\quad k\in R.
\end{equation*}

\end{document}
